\documentclass[OPS,authoryear,toc]{lsstdoc}
\input{meta}

% Package imports go here.

% Local commands go here.

%If you want glossaries
%\input{aglossary.tex}
%\makeglossaries

\title{Rubin Out of Hours Support}

% Optional subtitle
% \setDocSubtitle{A subtitle}

\author{%
Anastasia Alexov
}

\setDocRef{RTN-049}
\setDocUpstreamLocation{\url{https://github.com/lsst/rtn-049}}

\date{\vcsDate}

% Optional: name of the document's curator
% \setDocCurator{The Curator of this Document}

\setDocAbstract{%
Out of hours support proposal for the time period of Rubin Commissioning leading to and including Rubin Operations.  This document includes the IT DevOps team as well as Rubin Technical teams, who would need to provide emergency support during night time observing.  The primary purpose of the document is to outline the concept of out of hours support and propose compensation for this work for Rubin Chilean union employees.
}

% Change history defined here.
% Order: oldest first.
% Fields: VERSION, DATE, DESCRIPTION, OWNER NAME.
% See LPM-51 for version number policy.
\setDocChangeRecord{%
  \addtohist{1}{YYYY-MM-DD}{Unreleased.}{Anastasia Alexov}
}


\begin{document}

% Create the title page.
\maketitle
% Frequently for a technote we do not want a title page  uncomment this to remove the title page and changelog.
% use \mkshorttitle to remove the extra pages

% ADD CONTENT HERE
% You can also use the \input command to include several content files.

\appendix
% Include all the relevant bib files.
% https://lsst-texmf.lsst.io/lsstdoc.html#bibliographies
\section{References} \label{sec:bib}
\renewcommand{\refname}{} % Suppress default Bibliography section
\bibliography{local,lsst,lsst-dm,refs_ads,refs,books}

% Make sure lsst-texmf/bin/generateAcronyms.py is in your path
\section{Acronyms} \label{sec:acronyms}
\addtocounter{table}{-1}
\begin{longtable}{p{0.145\textwidth}p{0.8\textwidth}}\hline
\textbf{Acronym} & \textbf{Description}  \\\hline

A2 & Anastasia Alexov \\\hline
AURA & Association of Universities for Research in Astronomy \\\hline
CTIO & Cerro Tololo Inter-American Observatory \\\hline
ComCam & The commissioning camera is a single-raft, 9-CCD camera that will be installed in LSST during commissioning, before the final camera is ready. \\\hline
DM & Data Management \\\hline
DOE & Department of Energy \\\hline
EFD & Engineering and Facility Database \\\hline
IT & Information Technology \\\hline
LHN & long haul network \\\hline
M3 & tertiary mirror \\\hline
NCSA & National Center for Supercomputing Applications \\\hline
NFS & Network File System \\\hline
OPS & Operations \\\hline
RTN & Rubin Technical Note \\\hline
SLAC & SLAC National Accelerator Laboratory \\\hline
TBD & To Be Defined (Determined) \\\hline
UW & University of Washington \\\hline
\end{longtable}

% If you want glossary uncomment below -- comment out the two lines above
%\printglossaries





\end{document}
